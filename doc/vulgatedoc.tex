\documentclass[11pt,oneside,a4paper,article]{memoir}
\usepackage{fontspec}
\usepackage[unicode=true,xetex,colorlinks=true,linkcolor=blue,urlcolor=blue,bookmarksnumbered=true,bookmarksdepth=3]{hyperref}
\usepackage{bidi} % Must be last

%%%%%%%%%%%%%%%%%%%%%%%%%%%%%%%%%%%%%%%%%%%%%%%%%%%%%%%
%%%%%%%%%%%%%%%%%%% How to typeset %%%%%%%%%%%%%%%%%%%%
%%%%%%%%%%%%%%%%%%%%%%%%%%%%%%%%%%%%%%%%%%%%%%%%%%%%%%%

%% This file must be run through xelatex

%% The following fonts must be accessible to xelatex on
%% your system:
%%
%% Linux Libertine O
%%
%% On Fedora Linux, you can bring in these fonts like
%% so:
%%
%% sudo dnf install linux-libertine-fonts
%%



%%%%%%%%%%%%%%%%%%%%%%%%%%%%%%%%%%%%%%%%%%%%%%%%%%%%%%%
%%%%%%%%%%%%%%%%%%%% Configuration %%%%%%%%%%%%%%%%%%%%
%%%%%%%%%%%%%%%%%%%%%%%%%%%%%%%%%%%%%%%%%%%%%%%%%%%%%%%

%%% Fonts %%%
\setmainfont[Ligatures=TeX]{Linux Libertine O}

\newfontfamily{\mainnolig}{Linux Libertine O}
\newcommand{\q}{{\mainnolig '}}


%%% Page layout %%%
\settypeblocksize{247mm}{160mm}{*}
\setlrmargins{*}{*}{1}
\setulmargins{*}{*}{1}
\checkandfixthelayout

%%% Hyperref (Information in PDF) %%%
\hypersetup{
unicode=true,
pdfauthor={Claus Tøndering},
pdftitle={Generating jvulgate and VC}
}

%%% Lists %%%
\tightlists

%%% Allow extra space between words %%%
\sloppy


%%% Font matter %%%
\title{Generating jvulgate and VC}
\author{Claus Tøndering}
\date{17 January 2023}


\begin{document}
\maketitle


%%%%%%%%%%%%%%%%%%%%%%%%%%%%%%%%%%%%%%%%%%%%%%%%%%%%%%
%%%%%%%%%%%%%%%%%%%% Introduction %%%%%%%%%%%%%%%%%%%%
\chapter{Introduction}

This document gives a brief introduction to how to generate the jvulgate and VC Emdros databases for Bible
Online Learner on a Linux computer.

%%%%%%%%%%%%%%%%%%%%%%%%%%%%%%%%%%%%%%%%%%%%%%%%%%%%%%%
%%%%%%%%%%%%%%%%%%%% Prerequisites %%%%%%%%%%%%%%%%%%%%
\chapter{Prerequisites}

In order to execute this process, you need to \emph{recursively} clone the GitHub repository
\texttt{https://github.com/EzerIT/Vulgata.git}:

\vspace{1ex}

\quad\texttt{git clone --recursive https://github.com/EzerIT/Vulgata}

\vspace{1ex}



%%%%%%%%%%%%%%%%%%%%%%%%%%%%%%%%%%%%%%%%%%%%%%%%%%%%%%%%%%%%%
%%%%%%%%%%%%%%%%%%%% Generating jvulgate %%%%%%%%%%%%%%%%%%%%
\chapter{Generating jvulgate}

The ``jvulgate'' Emdros database contains most of the Latin NT in Jerome's Vulgate translation.

Executing the command ``make'' in the top directory generates the jvulgate Emdros database for Bible
OL.

The steps of the process are detailed in the following sections.


\section{Compiling \emph{parse\_latin\_nt}}

The C++ source code for \emph{parse\_latin\_nt} is compiled. A C++ compiler that supports the
``-std=c++20'' flag is required.


\section{Executing \emph{parse\_latin\_nt}}

\noindent \textbf{Inputs:}

\begin{itemize}
\item \texttt{proiel-treebank/latin-nt.xml}
\item \texttt{makewordlist/verber (konjugation).csv}
\item \texttt{makewordlist/Adjektiver (deklination + stamme).csv}
\item \texttt{makewordlist/substantiver (stamme).csv}
\end{itemize}

\noindent \textbf{Output:} \texttt{jvulgate.mql}

\vspace{1ex}

\noindent
The file \texttt{proiel-treebank/latin-nt.xml} contains the text to the New Testament. The
\texttt{*.csv} spreadsheets contains morphological information about each verb, adjective, and noun.
The spreadsheets were created by Mikkel Kølkær.

The \emph{parse\_latin\_nt} program uses these files to generate MQL code for the jvulgate database.

\section{Applying \texttt{jvulgate.mql}}

\noindent \textbf{Input:} \texttt{jvulgate.mql}

\noindent \textbf{Output:} The jvulgate Emdros database

\vspace{1ex}

\noindent
The ``mql'' command is executed and creates the jvulgate Emdros database from the MQL file
generated in the previous step.



%%%%%%%%%%%%%%%%%%%%%%%%%%%%%%%%%%%%%%%%%%%%%%%%%%%%%%%
%%%%%%%%%%%%%%%%%%%% Generating VC %%%%%%%%%%%%%%%%%%%%
\chapter{Generating VC}

The ``VC'' Emdros database contains chapter one of the beginner's Latin textbook \emph{Villa Claudiae}.

Executing the command ``make'' in the \texttt{villa\_claudiae} directory generates the VC Emdros
database for Bible OL.

The steps of the process are detailed in the following sections.


\section{Compiling \emph{parse\_latin\_vc}}

The C++ source code for \emph{parse\_latin\_vc} is compiled. A C++ compiler that supports the
``-std=c++20'' flag is required.


\section{Executing \emph{parse\_latin\_vc}}

\noindent \textbf{Inputs:}

\begin{itemize}
\item \texttt{Villa Claudiae.csv}
\item \texttt{../makewordlist/verber (konjugation).csv}
\item \texttt{../makewordlist/Adjektiver (deklination + stamme).csv}
\item \texttt{../makewordlist/substantiver (stamme).csv}
\end{itemize}

\noindent \textbf{Output:} \texttt{VC.mql}

\vspace{1ex}

\noindent
The file \texttt{Villa Claudiae.csv} contains the Latin text plus morphological information for
verbs, adjectives, and nouns not covered by the other .csv files. The
\texttt{*.csv} spreadsheets contains morphological information about each verb, adjective, and noun.
All the files were created by Mikkel Kølkær.

The \emph{parse\_latin\_vc} program uses these files to generate MQL code for the jvulgate database.

\section{Applying \texttt{VC.mql}}

\noindent \textbf{Input:} \texttt{VC.mql}

\noindent \textbf{Output:} The VC Emdros database

\vspace{1ex}

\noindent
The ``mql'' command is executed and creates the VC Emdros database from the MQL file generated in
the previous step.






%%%%%%%%%%%%%%%%%%%%%%%%%%%%%%%%%%%%%%%%%%%%%%%%%%%%%%%%%%%%%
%%%%%%%%%%%%%%%%%%%% Generating Lexicons %%%%%%%%%%%%%%%%%%%%
\chapter{Generating Lexicons}

There are currently no programs available for generating Latin lexicons.

\end{document}

% Local Variables:
% mode: latex
% ispell-dictionary: "british-ize"
% ispell-extra-args: ("--home-dir=/home/claus/Projects/BibleOL/techdoc")
% eval: (auto-fill-mode 1)
% End:
